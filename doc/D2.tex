%
% File acl2014.tex
%
% Contact: koller@ling.uni-potsdam.de, yusuke@nii.ac.jp
%%
%% Based on the style files for ACL-2013, which were, in turn,
%% Based on the style files for ACL-2012, which were, in turn,
%% based on the style files for ACL-2011, which were, in turn, 
%% based on the style files for ACL-2010, which were, in turn, 
%% based on the style files for ACL-IJCNLP-2009, which were, in turn,
%% based on the style files for EACL-2009 and IJCNLP-2008...

%% Based on the style files for EACL 2006 by 
%%e.agirre@ehu.es or Sergi.Balari@uab.es
%% and that of ACL 08 by Joakim Nivre and Noah Smith

\documentclass[11pt]{article}
\usepackage{acl2014}
\usepackage{times}
\usepackage{url}
\usepackage{latexsym}

%\setlength\titlebox{5cm}

% You can expand the titlebox if you need extra space
% to show all the authors. Please do not make the titlebox
% smaller than 5cm (the original size); we will check this
% in the camera-ready version and ask you to change it back.

\title{LING 573 Deliverable \#2 Report}
\tiny
\author{Haobo Gu \\
  \small University of Washington \\
  {\tt haobogu@uw.edu} \\\And
 Yuanhe Tian \\
 \small University of Washington \\
  {\tt yhtian@uw.edu}\\\And
 Weifeng Jin \\
 \small University of Washington \\
  {\tt wjin@uw.edu} \\\And
 Haotian Zhu \\
 \small University of Washington \\
  {\tt haz060@uw.edu}\\}

\date{}

\begin{document}
\maketitle
\begin{abstract}

This document discusses the baseline automatic summarization system our team built for the Deliverable \#2 of LING 573 in Spring 2018. The system takes input as a set of document sets and generates the summarization for each document set respectively. Our system has four major parts: data preprocessing, content selection, information ordering and content realization. These four parts were implemented separately and assembled together for submission.
\end{abstract}

\section{Introduction}

Automatic summarization is a traditional natural language processing task, whose aim is to shorten the input text without losing too much information in the context and create a abbreviated, informative and consistent summary. Generally speaking, there are two widely-used approaches for this problem, extraction-based summarization and abstraction-based summarization. In this baseline system, our team built a system upon the extraction-based method, which extracts sentences from the input document set without modifying the content of the sentences. Since this deliverable is a baseline system, we made our decision to use an unsupervised approach to assist with out extraction summary. Hence, there is no training and model construction required in this deliverable. Given the dataset documentation, our system can output the summary files individually without training process. We used \texttt{Python3} to implement our system.
\section{System Overview}

We employed the classic system structure mentioned in the lecture for the automatic summarization task, which consists three major parts, content selection, information ordering and content realization. Since the \texttt{AQUAINT} and \texttt{AQUAINT-2} Corpora have inconsistent data formatting, we add a data preprocessing part to prepare the data.  \\
\indent
The \texttt{data\_preprocessing} part takes one document specification (one \texttt{.xml} file) and the two corpora as input, and generate a \texttt{corpus} object, which includes \texttt{docsetList} and a token dictionary to record the frequency of each word occurrence in the input dataset. The \texttt{docsetList} object includes the document set ID, a token dictionary, a topic ID, a collection of summary texts generated by human beings and \texttt{documentCluster}. The \texttt{documentCluster} object is a list of document with its essential information, and each document has a list of sentences recorded.\\
\indent
The \texttt{content\_selection} part takes a document set as an input, and generates a list of sentences with their scores, computed by the LexRank algorithms. For each document set in the corpus, the part would generate a list of sentences individually.\\
\indent
The \texttt{information\_ordering} part takes a list of sentences, and returns a sorted list of sentences, using standards of chronology and other measures. \\
\indent
The \texttt{content\_realization} part takes the sorted list of sentences as input, and truncates the unnecessary sentences and low-rank sentences to generate an output summary for the corresponding document set. The output is a text file and for each document set in the corpus, the system would generate an individual summary text file for it.

\section{Approach}

As mentioned in the Introduction section, our system employs an unsupervised extraction approach and mainly uses the algorithms mentioned in the class. We used several different data structures, including list, dictionary and set to store the information extracted from the corpora and the detailed approach implementation for the three major parts would be discussed below and these approach is targeted to a single document set input.

\subsection{Content Selection}

Our team used the unsupervised LexRank model to calculate the score for each sentence in the document set. Suppose we have $m$ sentences and $n$ tokens in the document set. According to the occurrence of the token, we can build a feature vector of length $n$ for each sentence and generate a $m\times n$ matrix for the document set, in which each row represents a sentence's feature vector. In addition to that, we compute the cosine distance for each sentence pair and generate a new $m\times m$ matrix, in which the $ij^{th}$ entry represents the cosine similarity between the $i^{th}$ sentence and the $j^{th}$ sentence. Next, we have to convert the matrix to be a transition probability matrix, in which each row of the matrix would be normalized, and mark this matrix as $M$. \\
\indent
Generate a 1-D vector $P$,  of length $m$ to record the score of each sentence, in which the $i^{th}$ entry represents the score for the $i^{th}$ sentence. The final step is trying to converge the vector $P$ using the update rule $P_t = M^T P_{t-1}$ to converge. In our experiment, we set the tolerance to be 0.001 and the convergence speed very fast.

\subsection{Information Ordering}

Given a list of sentences selected by Content Selection procedure and salient scores associated, the baseline model of Information Ordering takes inter-document chronology and intra-document cohesion factors into account to generate a sorted list. Firstly, our model sorts the sentences by their chronological order, the dates of publication of the document where a particular sentence is extracted. The chronology information has been recorded in the data preprocessing part and would sort the sentences from different documents. Secondly, within the same document, sentences are ordered linearly by their positions in the document to impose cohesion. The position information has also been recorded using indices.\\
\indent
Finally, the Information Ordering part would produce a sorted list of sentences with respect to the chronology information of individual documents and the position information of the sentence in the context. 

\subsection{Content Realization}

We have tried two different approaches in our Content Realization module. The first method is an inline addition algorithm based on the sorted listed generated by the Information Ordering module. Beginning at the first element of the list, we add the sentence into our generation output and record the length of the whole output. Since the length of the summary document is limited to be less than 100 words, we use that to truncate unnecessary sentences and to move to the next one. This approach produces a very basic content realization result and does not have a satisfying performance on our end.\\
\indent
Next we tried to implement the Content Realization module with Integer Linear Programming, in which we try to optimize a target function. The input for the function is a set of sentences, and the output is a score for the set. In this case, this target function is the weighted average of the sum of all the selected sentences' scores and their diversity. The diversity factor is calculated from the bigrams generated from the selected sentences. The more bigrams the set of sentences has, the higher diversity it has. We also put several constraints for the optimization task. The first constraint is the whole length of sentences must be less than the limit. The second constraint is that if a certain bigram $j$ occurs in the summary, then the summary must contains a sentence having bigram $j$. The third constraint is that if a sentence $i$ is in the target set, then all the bigrams in the sentence would occur in the summary. Under these constraints, we use the Integer Linear Programming method to find the maximum value of the target function and generate the corresponding set of sentences to be our output.
\section{Results}

We tested our code on the \texttt{training} and \texttt{devtest} dataset and generated two different sets of summary texts. Although our current baseline model does not require additional time for model training, it still takes a while to compute the score matrices and optimize the target function in the Content Selection and Content Realization module. After that, we used the \texttt{ROUGE} script to compare the generated summaries and the human summaries (as gold standard), and we get the following results.
\begin{table}[h!]
\centering
\caption{ROUGE Results for \texttt{devtest}}
\begin{tabular}{|l|l|l|l|}
\hline
ROUGE   & Recall  & Precision & F-score \\ \hline
ROUGE-1 & 0.20038 & 0.25005   & 0.22145 \\ \hline
ROUGE-2 & 0.04347 & 0.05310   & 0.04759 \\ \hline
ROUGE-3 & 0.01218 & 0.01495   & 0.01336 \\ \hline
ROUGE-4 & 0.00356 & 0.00433   & 0.00389 \\ \hline
\end{tabular}
\end{table}
\begin{table}[h!]
\centering
\caption{ROUGE Results for \texttt{training}}
\begin{tabular}{|l|l|l|l|}
\hline
ROUGE   & Recall  & Precision & F-score \\ \hline
ROUGE-1 & 0.20231 & 0.28003   & 0.23382 \\ \hline
ROUGE-2 & 0.05260 & 0.07219   & 0.06062 \\ \hline
ROUGE-3 & 0.01885 & 0.02640   & 0.02191 \\ \hline
ROUGE-4 & 0.00847 & 0.01224   & 0.00997 \\ \hline
\end{tabular}
\end{table}

\section{Discussion}
Frankly speaking, from the results shown in the above tables, this is not a satisfying system for the task. The performance indicators, namely the recall, precision and F-score for each  \texttt{ROUGE} categories, are not good enough to deliver a readable and comprehensible machine-generated summary. As stated in the specification of this deliverable, this is a baseline automatic document summarization system focused primarily on connectivity and efficiency and we think our project delivers the goal and makes the system work without incurring significant problems.\\
\indent
For the baseline system, we employed an unsupervised, extraction-based algorithm in all three major parts of the summary generation. It does help our team to generate the summaries relatively quick and without training model, there is no risk for overfitting the training dataset. However, without a proper training process, our model might lose lots of its accuracy and credibility. In the future, our team might want to use a supervised algorithm to perform the training process first and then uses a trained model to generate summaries from the \texttt{devtest} and \texttt{evaltest} datasets. 

\section{Conclusion}

This deliverable is our first team project for this course, and also our first trial towards building an automatic document summarization system. All of our four team members work well together, with each member focusing on a specific module and we connect the system together in the end to make it work. Our project delivers a baseline system focused on unsupervised, extraction-based and connectivity-focused summarization system. The performance of our system does left a lot improving space for us to work in the future and we want to introduce several supervised learning algorithms to make the system better.

\begin{thebibliography}{}
\bibitem[\protect\citename{Galanis, Lampouras and Androutsopoulos}2013]{Galanis:12}
Dimitrios~Galanis, Gerasimos~Lampouras and Ion~Androutsopoulos.
\newblock 2013.
\newblock {\em Extractive Multi-Document Summarization with Integer Linear Programming and Support Vector Regression.}
\newblock{Proceedings of COLING 2012, 2012: 911-926.}

\bibitem[\protect\citename{Georgi}2018]{Georgi:18}
Ryan~A.Georgi
\newblock 2018.
\newblock {\em Lecture of Content Selection and Information Ordering.}
\newblock{in the course LING 573 Spring 2018 in the University of Washington.}

\end{thebibliography}

\end{document}
